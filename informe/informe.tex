\documentclass[a4paper,spanish]{article}

%\usepackage[spanish]{babel}
\usepackage[utf8x]{inputenc}
\usepackage{babel}
\usepackage{tikz}
\usetikzlibrary{babel}
\usepackage{amsmath}
\usepackage{graphicx}
\usepackage[colorinlistoftodos]{todonotes}
\usepackage{comment}
\usepackage{float}
\usepackage{subcaption}
\usepackage{verbatim}
\usepackage[active,tightpage]{preview}
\PreviewEnvironment{tikzpicture}


\title{Practica Hough - Rectas - IPDI 2° C 2015}
\author{Emilio Gasco}

\begin{document}
\maketitle

%\begin{abstract}
%\end{abstract}
\renewcommand{\thesection}{Ejercicio \arabic{section}}
\renewcommand{\thesubsection}{\alph{subsection}}
\renewcommand{\thesubsubsection}{\roman{subsubsection}}
\newcommand{\myss}[3]{
    \begin{subfigure}[b]{#1\textwidth}
        \includegraphics[width=\textwidth]{img/#2}
        \caption{#3}
        \label{fig:#2}
    \end{subfigure}
}

\section{Ecualizacion + Hough + canny standard}

* Dificil setear umbrales que que funcionen bien durante tood el video
*  Lineas horizontales detectan innecesariamente.
\begin{verbatim}
max_frame=1431;

for i=1:max_frame
    fname = strcat('camino1/frame_',int2str(i),'.png');
    im = imread(fname);
    im=ecualizar_histograma(im);
    [ nim,A,mix,borders  ] = ld_hough( im, 'canny',50,130,pi/180,100);
    fname = strcat('output/camino1/frame_',int2str(i),'.png');
    imwrite(nim,fname);
    imshow(nim,[]);
end
\end{verbatim}


\begin{tikzpicture}[y=-1cm]
  \draw (0,2) -- (10,2) node[below] {$x$};
  \draw[->] (0,0) -- (0,5) node[left] {$y$};
  \draw[->] (5,0) -- (5,5) node[left] {$y'$};
\end{tikzpicture}

\begin{itemize}
\item La probablilidad de que suma de gradiente tiende a seleccionar malos pares cuando el umbral de hough esta muy alejado del maximo. 
\item Si la linea "mala" es de bajos votos, el echo que el umbral se calcule como 0.9 votos(linea) vuelve al algortimo inestable. 
\end{itemize}



\end{document}
