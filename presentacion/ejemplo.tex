\begin{frame}
\frametitle{Introducción - Problema}
\begin{itemize}
\item Supongamos que un ISP desea analizar los datos acerca de los paquetes que circulan por la red en tiempo real.
\item Por ejemplo, interesa analizar la frecuencia con la que llegan los paquetes de las distintas IPs para así, entre otras cosas, poder detectar y evitar ataques de Denial of Service. O quizá resulta relevante determinar si ciertos nodos de la red están saturados. 
\item Queda claro que la información llega de forma constante y en grandes volúmenes de datos a la base de datos del sistema, puesto que constantemente están transitando numerosos paquetes por la red.
\item No es estrictamente necesario obtener resultados exactos: basta con una aproximación. En efecto, se busca detectar patrones o comportamientos estadísticos más que hacer un análisis exhaustivo y exacto de los datos.
\end{itemize}
\end{frame}

\begin{frame}
\frametitle{Introducción - Problema (Cont.)}
\begin{block}{Pregunta 1}
¿Podríamos con un sistema de administración de base de datos tradicional realizar consultas para obtener los resultados deseados en tiempo real? ¿Por qué?
\end{block}
\begin{block}{Pregunta 2}
¿Y con uno que modelara los datos como un flujo y permitiera consultas que computen el resultado a medida que los datos llegan?
\end{block}
\end{frame}

