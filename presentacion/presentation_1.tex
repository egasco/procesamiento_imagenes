%%%%%%%%%%%%%%%%%%%%%%%%%%%%%%%%%%%%%%%%%
% Beamer Presentation
% LaTeX Template
% Version 1.0 (10/11/12)
%
% This template has been downloaded from:
% http://www.LaTeXTemplates.com
%
% License:
% CC BY-NC-SA 3.0 (http://creativecommons.org/licenses/by-nc-sa/3.0/)
%
%%%%%%%%%%%%%%%%%%%%%%%%%%%%%%%%%%%%%%%%%

%----------------------------------------------------------------------------------------
%	PACKAGES AND THEMES
%----------------------------------------------------------------------------------------

\documentclass{beamer}
\usepackage[utf8]{inputenc}
\usepackage[spanish]{babel}


\mode<presentation> {

% The Beamer class comes with a number of default slide themes
% which change the colors and layouts of slides. Below this is a list
% of all the themes, uncomment each in turn to see what they look like.

%\usetheme{default}
%\usetheme{AnnArbor}
%\usetheme{Antibes}
%\usetheme{Bergen}
%\usetheme{Berkeley}
%\usetheme{Berlin}
%\usetheme{Boadilla}
%\usetheme{CambridgeUS}
%\usetheme{Copenhagen}
%\usetheme{Darmstadt}
%\usetheme{Dresden}
%\usetheme{Frankfurt}
%\usetheme{Goettingen}
%\usetheme{Hannover}
%\usetheme{Ilmenau}
%\usetheme{JuanLesPins}
%\usetheme{Luebeck}
\usetheme{Madrid}
%\usetheme{Malmoe}
%\usetheme{Marburg}
%\usetheme{Montpellier}
%\usetheme{PaloAlto}
%\usetheme{Pittsburgh}
%\usetheme{Rochester}
%\usetheme{Singapore}
%\usetheme{Szeged}
%\usetheme{Warsaw}

% As well as themes, the Beamer class has a number of color themes
% for any slide theme. Uncomment each of these in turn to see how it
% changes the colors of your current slide theme.

%\usecolortheme{albatross}
\usecolortheme{beaver}
%\usecolortheme{beetle}
%\usecolortheme{crane}
%\usecolortheme{dolphin}
%\usecolortheme{dove}
%\usecolortheme{fly}
%\usecolortheme{lily}
%\usecolortheme{orchid}
%\usecolortheme{rose}
%\usecolortheme{seagull}
%\usecolortheme{seahorse}
%\usecolortheme{whale}
%\usecolortheme{wolverine}

%\setbeamertemplate{footline} % To remove the footer line in all slides uncomment this line
%\setbeamertemplate{footline}[page number] % To replace the footer line in all slides with a simple slide count uncomment this line

%\setbeamertemplate{navigation symbols}{} % To remove the navigation symbols from the bottom of all slides uncomment this line
}

\usepackage{graphicx} % Allows including images
\usepackage{booktabs} % Allows the use of \toprule, \midrule and \bottomrule in tables

%----------------------------------------------------------------------------------------
%	TITLE PAGE
%----------------------------------------------------------------------------------------

\title[Hough Lineal]{Hough Lineal} % The short title appears at the bottom of every slide, the full title is only on the title page

\author{Emilio Gasco} % Your name
\institute[FCEyN] % Your institution as it will appear on the bottom of every slide, may be shorthand to save space
{
Facultad de Ciencias Exactas y Naturales\\
Universidad de Buenos Aires\\ % Your institution for the title page
\medskip
%\textit{john@smith.com} % Your email address
}
\date{\today} % Date, can be changed to a custom date

\begin{document}

\begin{frame}
\titlepage % Print the title page as the first slide
\end{frame}

\begin{frame}
\frametitle{Overview} % Table of contents slide, comment this block out to remove it
\tableofcontents % Throughout your presentation, if you choose to use \section{} and \subsection{} commands, these will automatically be printed on this slide as an overview of your presentation
\end{frame}

%----------------------------------------------------------------------------------------
%	PRESENTATION SLIDES
%----------------------------------------------------------------------------------------

%------------------------------------------------
\section{Introducción} 
\begin{frame}
\frametitle{Introducción - Problema}
\begin{itemize}
\item Supongamos que un ISP desea analizar los datos acerca de los paquetes que circulan por la red en tiempo real.
\item Por ejemplo, interesa analizar la frecuencia con la que llegan los paquetes de las distintas IPs para así, entre otras cosas, poder detectar y evitar ataques de Denial of Service. O quizá resulta relevante determinar si ciertos nodos de la red están saturados. 
\item Queda claro que la información llega de forma constante y en grandes volúmenes de datos a la base de datos del sistema, puesto que constantemente están transitando numerosos paquetes por la red.
\item No es estrictamente necesario obtener resultados exactos: basta con una aproximación. En efecto, se busca detectar patrones o comportamientos estadísticos más que hacer un análisis exhaustivo y exacto de los datos.
\end{itemize}
\end{frame}

\begin{frame}
\frametitle{Introducción - Problema (Cont.)}
\begin{block}{Pregunta 1}
¿Podríamos con un sistema de administración de base de datos tradicional realizar consultas para obtener los resultados deseados en tiempo real? ¿Por qué?
\end{block}
\begin{block}{Pregunta 2}
¿Y con uno que modelara los datos como un flujo y permitiera consultas que computen el resultado a medida que los datos llegan?
\end{block}
\end{frame}



\begin{frame}
\frametitle{Introducción}
\begin{itemize}

\item Recientemente, han surgido aplicaciones donde la información se modela mejor como un flujo de datos transitorio que como relaciones persistentes.
\item A pesar de que los elementos de datos suelen ser tuplas relacionales, el hecho de que arriben en un flujo constante que no está acotado ni temporal ni espacialmente hace que los sistemas relaciones tradicionales no sean adecuados:
\begin{itemize}
\item No están optimizados para una carga rápida y constante de datos.
\item No soportan nativamente consultas continuas. Es decir, consultas que vayan actualizando el resultado a medida que llegan nuevos datos.
\item Están enfocadas en proveer consultas precisas mediante planes de consulta estables, pero cuando se quiere trabajar con flujos de datos no acotados suele ser necesario trabajar con aproximaciones y adapatarse a variaciones del flujo.
\end{itemize}
\end{itemize}
\end{frame}


\section{Modelo de Datos para Stream Databases}

\begin{frame}
\frametitle{Modelo de flujo de datos}
\begin{block}{}
El \textbf{data stream model}, o el modelo de flujo de datos, surge para modelar situaciones donde la información, a pesar de llegar como tuplas relacionales, no lo hace en sólo forma de relaciones persistentes (como en el modelo de datos relacional), sino que posee la característica de arribar de forma continua en múltiples flujos, veloces, variables en el tiempo, impredecibles y potencialmente no acotados.\
\end{block}
\end{frame}

\begin{frame}
\frametitle{Modelo de flujo de datos (Cont.)}
Los datos de entrada deben ser operados continuamente a medida que arriban. La información no está disponible para su acceso aleatorio en disco o memoria. Algunas diferencia con los DBMS tradicionales son:
\begin{itemize}
\item Los datos llegan por la red.
\item No se tiene control sobre el orden en que llegan los datos, sobre todo cuando se procesan múltiples flujos de datos.
\item Los flujos de datos son, en principio, de tamaño ilimitado.
\item Una vez que un elemento del flujo es procesado, se descarta o se archiva, por lo que potencialmente el dato deja de estar disponible para futuras consultas.
\end{itemize} 
\end{frame}

\begin{frame}
\frametitle{Consultas One-Time vs. Consultas Continuas}

\begin{block}{Consultas One-Time}
Son aquellas que se utilizan en los sistemas de administración de bases de datos tradicionales. Se evalúan una única vez (en el sentido de que la consulta no persiste activa en el tiempo) sobre lo que podría considerarse un \textbf{snapshot} del conjunto de datos para un momento dado.
\end{block}

\begin{block}{Consultas Continuas}
Son las que nos interesan en este modelo. No están soportadas en administradores tradicionales. Se ejecutan continuamente a lo largo del tiempo y actualizan su resultado, siempre teniendo el cuenta el flujo visto. Dadas las limitaciones en el manejo de flujo de datos, no suele requerirse que devuelvan un resultado exacto: una aproximación suele ser suficiente.
\end{block}

\end{frame}

\begin{frame}
\frametitle{Consultas Predefinidas vs. Consultas Ad-Hoc}
\begin{block}{Consultas Predefinidas}
Se diagraman y se informan al DSMS antes de la llegada de datos.
\end{block}

\begin{block}{Consultas Ad-Hoc}
Se definen y ejecutan una vez iniciado el flujo de datos. Este tipo de consultas complica el diseño de los DSMS por que dificulta la optimización de la consulta y porque, en algunos casos, puede ser necesario referenciar datos ya descartados para responderla correctamente.
\end{block}
\end{frame}
% \begin{frame}

% Aca podemos agregar algun ejemplo( el que esta en el paper para bajar las cosas un poco  a tierra 

% \end{frame}

\section{Procesamiento de  consultas en Stream Databases}
\subsection{Desafíos para el procesamiento de consultas}
\begin{frame}
\frametitle{Desafíos para el procesamiento de consultas en el modelo}
\begin{itemize}
\item Requerimientos de memoria no acotados.
\item Respuestas aproximadas a Consultas.
\item Operaciones Bloqueantes.
\item Referencias a datos pasados.
\end{itemize}
\end{frame}



\begin{frame}
\frametitle{Requerimientos de Memoria no acotados}
\begin{itemize}
\item El flujo de datos no está acotado en tamaño.
\item Se requiere respuestas a las consultas en tiempo real, por lo cual se desea un tiempo de cómputo bajo.
\item Se necesitan algoritmos que minimicen el acceso a disco y puedan operar completamente en memoria física.
\item Los algoritmos existentes no son lo suficientemente eficientes para satisfacer estas condiciones.
\item Es un problema abierto si se quiere obtener una solución exacta.
\end{itemize}
\end{frame}

\begin{frame}
\frametitle{Respuestas aproximadas a Consultas}
\begin{itemize}
\item Las limitaciones de memoria y las restricciones temporales imposibilitan el desarrollo de algoritmos que provean soluciones exactas a las consultas.
\item Suelen aceptarse soluciones aproximadas. Para ello se utilizan diversas técnicas (cada una con su propia definición de aproximación):
\begin{itemize}
\item Sketch
\item Sampling aleatorio
\item Histogramas
\item Wavelets
\item Ventanas Deslizantes
\end{itemize}
\end{itemize}
\end{frame}

\begin{frame}
\frametitle{Aproximación - Ventanas Deslizantes}
\begin{itemize}
\item Consiste en evaluar una consulta no sobre la totalidad de los datos obtenidos del flujo sino sobre una ventana temporal de datos correspondiente con aquellos recientes.
\item Está bien definida y tiene una semántica de aproximación clara: el usuario es consciente de lo que se está dejando de lado al producir una respuesta.
\item Es determinística.
\item Enfatiza los datos recientes, que suelen ser los más importantes en las consultas de este modelo.
\end{itemize}
\end{frame}

\begin{frame}
\frametitle{Aproximación - Ventanas Deslizantes - Timestamps}
Las ventanas deslizantes requieren una noción de orden sobre el flujo de datos. Para ello, se recurre al uso de timestamps.
\begin{block}{Timestamp Implícito}
Se corresponden con el orden de llegada de los datos y suelen ser utilizadas cuando el momento exacto de tiempo asociado a una tupla no es importante, sino la consideración acerca de si es $reciente$ o $antiguo$.
\end{block}

\begin{block}{Timestamp Explícito}
 Se definen explícitamente por cada elemento y suelen utilizarse cuando el momento exacto de la llegada de un dato es de importancia para el significado de la tupla. El gran inconveniente que presentan es que el timestamp asociado no necesariamente se corresponda con el orden de llegada de los datos.
\end{block}
\end{frame}

\begin{frame}
\frametitle{Aproximación - Ventanas Deslizantes - Timestamps (Cont.)}
Para asignar timestamps al resultado de una operación sobre dos flujos:
\begin{itemize}
\item No dar garantías sobre el orden de salida de los datos, sino simplemente en asumir que lo más probable es que las tuplas que arriban más temprano serán también aquellas en las que el operador actuará de manera más temprana.
\item Explicitar como parte de la consulta qué timestamps asignarle a las tuplas del resultado, por ejemplo, priorizando los flujos y correspondiendole a cada resultado el timestamp del flujo más prioritario.
\end{itemize}
\end{frame}

\begin{frame}
\frametitle{Aproximación - Procesamiento por lote,  muestreo y sinopsis}
Supongamos que una consulta sobre un flujo de datos se responde usando una estructura de datos que puede ser mantenida incrementalmente. Dicha estructura tiene dos operaciones:
\begin{itemize}
\item actualizar$($tupla$)$: actualiza la estructura con una nueva tupla.
\item computarRespuesta$()$: computa y produce o actualiza los resultados de la consulta.
\end{itemize}
Idealmente, ambas se pueden procesar rápidamente y proveen una respuesta en tiempo real. Pero...
\end{frame}

\begin{frame}
\frametitle{Aproximación - Procesamiento por lote}
\begin{itemize}
\item actualizar$($tupla$)$ es rápida pero computarRespuesta$()$ no lo es.
\item En vez de producir constantemente una respuesta actualizada, se actualizan las estructuras de datos a medida que llegan los datos, no se descarta ninguno, y se actualiza la respuesta periódicamente.
\item La respuesta es aproximada en el sentido de que no refleja los últimos datos, aunque el resultado es exacto (para algún momento pasado). 
\item Un algoritmo que tal vez no pueda mantener el ritmo de procesamiento cuando hay un pico en el flujo de datos tal vez sí pueda mantenerse estable lidiando con los datos haciendo uso de esta técnica.
\end{itemize}
\end{frame}

\begin{frame}
\frametitle{Aproximación - Muestreo y Sinopsis}
\textbf{Muestreo}
\begin{itemize}
\item computarRespuesta$()$ es rápida pero actualizar$($tupla$)$ es lenta.
\item Resulta inútil hacer uso de todos los datos, puesto que arriban más rápido que lo que tardan en ser procesados.
\item Se toma una muestra de los datos, ignorando completamente algunas tuplas y se genera una respuesta en base a los mismos.
\item La respuesta es aproximada en el sentido de que no se tomaron en cuenta todos los datos de entrada.
\item En muchos casos no es posible garantizar un error acotado.
\end{itemize}
\textbf{Sinopsis}
\begin{itemize}
\item Ambas operaciones son lentas.
\item Se diseña una estructura de datos aproximada que mantiene una sinopsis o sketch de los datos en pos de una representación exacta.
\item Minimiza el cómputo necesario por tupla.
\end{itemize}
\end{frame}

\begin{frame}
\frametitle{Operadores Bloqueantes}
\begin{itemize}
\item Son aquellas operaciones que no pueden producir la primer tupla del resultado hasta que se hayan visto todos los datos.
\begin{itemize}
\item Operaciones de agregación.
\item Operaciones de ordenamiento.
\end{itemize}
\item Suelen ser operaciones comunes y, por lo tanto, deseables en un sistema de administración de bases de datos que soporte este modelo.
\item Sin embargo, los flujos de datos son potencialmente infinitos...
\end{itemize}
\end{frame}

\begin{frame}
\frametitle{Operadores Bloqueantes (Cont.)}
\begin{itemize}
\item Si la operación se encuentra en la raíz del árbol de ejecución:
\begin{itemize}
\item Devolver un flujo de datos que se va actualizando a medida que arriban datos nuevos del flujo si el resultado es pequeño.
\item Mantener una estructura de datos si el resultado es grande.
\end{itemize}
\item Si la operación se encuentra en un nodo interno del árbol de ejecución:
\begin{itemize}
\item El problema es que los resultados de estos operadores varían conforme llegan nuevos datos.\\
\item Propuesta 1: reemplazar las operaciones bloqueantes por otras no bloqueantes que sean análogas pero aproximadas (por ejemplo usar juggle para hacer un ordenamiento, ordenando localmente los datos de manera no bloqueante).
\item Propuesta 2: aumentar el flujo de datos con aserciones sobre los datos por venir. Estas aserciones se llaman \textbf{punctuations} o puntuaciones y nos permiten tomar decisiones sobre los datos ya procesados.
\end{itemize}
\end{itemize}
\end{frame}

\begin{frame}
\frametitle{Referencias a datos pasados}
\begin{itemize}
\item Una de las características del Data Stream Model es que una vez que un elemento ya pasó por el flujo, entones no necesariamente puede ser visitado nuevamente.
\item Algunas consultas Ad-hoc no se podrían responder con exactitud dado que algunos datos ya han sido descartados.
\item Una solución simple consiste en que las consultas Ad-hoc solamente referencien datos por venir.
\item Una solución mas ambiciosa consiste en mantener un resumen de cada flujo de datos y así permitir consultas Ad-hoc que referencien a datos pasados y poder dar una mejor aproximación.
\end{itemize}
\end{frame}

\subsection{STREAM - Stanford Stream Data Manager}

\begin{frame}
\frametitle{STREAM - Stanford Stream Data Manager}
\begin{itemize}
\item Diseñado e implementando en Stanford.
\item Desarrollo de un sistema de administración de bases de datos basados en el Data Stream Model.
\item Implementa mecanismos para realizar consultas continuas, técnicas de aproximación y estructuras de sinopsis.
\end{itemize}
\end{frame}

\begin{frame}
\frametitle{STREAM - Stanford Stream Data Manager (Cont.)}
\begin{itemize}
\item Utiliza una modificación de SQL como lenguaje de consulta:
\begin{itemize}
\item Se agregó la posibilidad referenciar flujos de datos en el FROM.
\item Se extendió la declaratividad del lenguaje para indicar ventanas deslizantes. Se indica:
\begin{itemize}
\item especificación de la ventana encerrada en corchetes.
\item una partición opcional de los datos en distintos grupos, mantenidos en distintas ventanas.
\item tamaño de ventana, expresado en unidades físicas (como número de elementos) o en unidades lógicas (como el rango temporal de la ventana).
\item predicado de filtro opcional.
\end{itemize}
\end{itemize} 
\end{itemize}
\end{frame}

\begin{frame}[fragile]
\frametitle{STREAM - Stanford Stream Data Manager (Cont.)}
Ejemplo de consulta en STREAM:\\
\textit{Obtener el tiempo de duración promedio de una llamada considerando solamente las diez llamadas más recientes de larga distancia por usuario.} 

\begin{verbatim}
SELECT AVG(S.minutes)
FROM Calls S [PARTITION BY S.customer id
             ROWS 10 PRECEDING
             WHERE S.type = ’Long Distance’]
\end{verbatim}
\begin{itemize}
\item Cláusula RECENT : el sistema utiliza un orden propio para producir la ventana.
\item Cláusula PRECEDING : se utiliza cuando la unidad de la ventana es lógica o física, y expresa que el ordenamiento debe seguir el orden dado por los timestamps.
\end{itemize}
\end{frame}

\begin{frame}
\frametitle{STREAM - Stanford Stream Data Manager (Cont.)}
\begin{itemize}
\item En STREAM, un plan de ejecución consiste de operadores conectados por colas. Además, cada operador puede mantener un estado intermedio haciendo uso de sinopsis en alguna estructura de datos, para calcular, por ejemplo, una ventana deslizante.
\item La memoria del sistema se particiona dinámicamente entre las sinopsis y las colas de los planes de ejecución, al igual que para los buffers utilizados para manejar los flujos que arriban y un cache para los datos que se encuentran en disco.
\item Un scheduler es el encargado de repartir el tiempo de ejecución entre los operadores. Durante la ejecución, un operador lee los datos de las colas de entrada, actualiza las estructuras de sinopsis que mantiene y escribe los resultados en sus colas de salida.
\end{itemize}
\end{frame}

\begin{frame}
\frametitle{STREAM - Stanford Stream Data Manager (Cont.)}
\begin{itemize}
\item Durante el ciclo de vida de una operación, muchos de los parámetros de la consulta como la característica del flujo, la tasa de datos o a cantidad de consultas que están corriendo de manera concurrente puede variar de manera considerable. Es por esta razón, que todos los operadores de STREAM son \textbf{adaptativos}.
\item STREAM sólo encaró el problema de la adaptatividad en relación con la memoria disponible.
\item En STREAM cada operador \textbf{maximiza la exactitud de su respuesta adaptándose a la cantidad de la memoria que tiene disponible para su uso} y maneja los cambios dinámicos que pueden surgir en el tamaño de su memoria En pocas palabras, a mayor memoria disponible, la aproximación del resultado será mejor.
\end{itemize}
\end{frame}


\begin{frame}
\frametitle{Bibliografía}
\begin{itemize}
\item \textbf{Models And Issues in Data Strean Systems}; Babcock, Babu, Datar, Motwani, Widom.
\end{itemize}
\end{frame}

% \section{Ejemplos template}
% \subsection{Subsection Example} % A subsection can be created just before a set of slides with a common theme to further break down your presentation into chunks

% \begin{frame}
% \frametitle{Paragraphs of Text}
% Sed iaculis dapibus gravida. Morbi sed tortor erat, nec interdum arcu. Sed id lorem lectus. Quisque viverra augue id sem ornare non aliquam nibh tristique. Aenean in ligula nisl. Nulla sed tellus ipsum. Donec vestibulum ligula non lorem vulputate fermentum accumsan neque mollis.\\~\\

% Sed diam enim, sagittis nec condimentum sit amet, ullamcorper sit amet libero. Aliquam vel dui orci, a porta odio. Nullam id suscipit ipsum. Aenean lobortis commodo sem, ut commodo leo gravida vitae. Pellentesque vehicula ante iaculis arcu pretium rutrum eget sit amet purus. Integer ornare nulla quis neque ultrices lobortis. Vestibulum ultrices tincidunt libero, quis commodo erat ullamcorper id.
% \end{frame}

% %------------------------------------------------

% \begin{frame}
% \frametitle{Bullet Points}
% \begin{itemize}
% \item Lorem ipsum dolor sit amet, consectetur adipiscing elit
% \item Aliquam blandit faucibus nisi, sit amet dapibus enim tempus eu
% \item Nulla commodo, erat quis gravida posuere, elit lacus lobortis est, quis porttitor odio mauris at libero
% \item Nam cursus est eget velit posuere pellentesque
% \item Vestibulum faucibus velit a augue condimentum quis convallis nulla gravida
% \end{itemize}
% \end{frame}

% %------------------------------------------------

% \begin{frame}
% \frametitle{Blocks of Highlighted Text}
% \begin{block}{Block 1}
% Lorem ipsum dolor sit amet, consectetur adipiscing elit. Integer lectus nisl, ultricies in feugiat rutrum, porttitor sit amet augue. Aliquam ut tortor mauris. Sed volutpat ante purus, quis accumsan dolor.
% \end{block}

% \begin{block}{Block 2}
% Pellentesque sed tellus purus. Class aptent taciti sociosqu ad litora torquent per conubia nostra, per inceptos himenaeos. Vestibulum quis magna at risus dictum tempor eu vitae velit.
% \end{block}

% \begin{block}{Block 3}
% Suspendisse tincidunt sagittis gravida. Curabitur condimentum, enim sed venenatis rutrum, ipsum neque consectetur orci, sed blandit justo nisi ac lacus.
% \end{block}
% \end{frame}

% %------------------------------------------------

% \begin{frame}
% \frametitle{Multiple Columns}
% \begin{columns}[c] % The "c" option specifies centered vertical alignment while the "t" option is used for top vertical alignment

% \column{.45\textwidth} % Left column and width
% \textbf{Heading}
% \begin{enumerate}
% \item Statement
% \item Explanation
% \item Example
% \end{enumerate}

% \column{.5\textwidth} % Right column and width
% Lorem ipsum dolor sit amet, consectetur adipiscing elit. Integer lectus nisl, ultricies in feugiat rutrum, porttitor sit amet augue. Aliquam ut tortor mauris. Sed volutpat ante purus, quis accumsan dolor.

% \end{columns}
% \end{frame}

% %------------------------------------------------
% \section{Second Section}
% %------------------------------------------------

% \begin{frame}
% \frametitle{Table}
% \begin{table}
% \begin{tabular}{l l l}
% \toprule
% \textbf{Treatments} & \textbf{Response 1} & \textbf{Response 2}\\
% \midrule
% Treatment 1 & 0.0003262 & 0.562 \\
% Treatment 2 & 0.0015681 & 0.910 \\
% Treatment 3 & 0.0009271 & 0.296 \\
% \bottomrule
% \end{tabular}
% \caption{Table caption}
% \end{table}
% \end{frame}

% %------------------------------------------------

% \begin{frame}
% \frametitle{Theorem}
% \begin{theorem}[Mass--energy equivalence]
% $E = mc^2$
% \end{theorem}
% \end{frame}

% %------------------------------------------------

% \begin{frame}[fragile] % Need to use the fragile option when verbatim is used in the slide
% \frametitle{Verbatim}
% \begin{example}[Theorem Slide Code]
% \begin{verbatim}
% \begin{frame}
% \frametitle{Theorem}
% \begin{theorem}[Mass--energy equivalence]
% $E = mc^2$
% \end{theorem}
% \end{frame}\end{verbatim}
% \end{example}
% \end{frame}

% %------------------------------------------------

% \begin{frame}
% \frametitle{Figure}
% Uncomment the code on this slide to include your own image from the same directory as the template .TeX file.
% %\begin{figure}
% %\includegraphics[width=0.8\linewidth]{test}
% %\end{figure}
% \end{frame}

% %------------------------------------------------

% \begin{frame}[fragile] % Need to use the fragile option when verbatim is used in the slide
% \frametitle{Citation}
% An example of the \verb|\cite| command to cite within the presentation:\\~

% This statement requires citation \cite{p1}.
% \end{frame}

% %------------------------------------------------

% \begin{frame}
% \frametitle{References}
% \footnotesize{
% \begin{thebibliography}{99} % Beamer does not support BibTeX so references must be inserted manually as below
% \bibitem[Smith, 2012]{p1} John Smith (2012)
% \newblock Title of the publication
% \newblock \emph{Journal Name} 12(3), 45 -- 678.
% \end{thebibliography}
% }
% \end{frame}

% %------------------------------------------------

% \begin{frame}
% \Huge{\centerline{The End}}
% \end{frame}

%----------------------------------------------------------------------------------------

\end{document} 
